\documentclass{ks-pset}

\usepackage{ks-cs}

\title{Homework 1}
\author{Kye Shi}
\date{2022 January 27}

\begin{document}

\begin{itemize}
  \item Please read \emph{Handout 1:  Good Writing, Induction, Extremality, and
    Greed} before embarking on this assignment.  It's a review of important
    concepts and techniques that you'll use on this assignment.  It also
    explains class expectations on writing style.

  \item Each problem must be typeset using \LaTeX.  You're welcome to start
    with the \LaTeX{} source files posted on Piazza and modify them.

  \item At the end of each problem, please include a brief self-assessment.  If
    you know that your solution is not quite complete or correct, let us know
    that you recognize that and where you believe that the issue arises.  If
    you're confident in your solution, let us know that as well! Generally,
    just one sentence suffices here.  Part of your score will be based on an
    accurate self-assessment of your work.

  \item After Tuesday's class (January 21), you're ready to do Problems 1, 2,
    and 4.  Starting early on these problem sets (or at least thinking about how
    to solve them) is a great idea for this class! Problem 3 is based on
    material in Thursday's class.

  \item Most problem sets (including this one) have an optional bonus problem.
    While the required problems are intended to develop \emph{proficiency} with
    the material, the optional problems will help you achieve deeper
    \emph{mastery} of the material.

  The required problems are worth a total of 100 points and that's all that's
  required for a perfect score.  However, it's possible to get more than 100
  points by submitting a correct solution to the bonus mastery problem and
  those points can positively impact your course grade.
\end{itemize}

\newpage

\begin{problem}[Positively Beaming, 20]
  You are the manager of a research facility that provides high-powered X-ray
  beam access for researchers interested in using these beams for their
  experiments.  Research groups reserve time slots with the facility months in
  advance lasting one or more days, with many research groups present at the
  facility at any one time.  We will say $n$ groups are signed up for beam time
  over the next $D$ days, with group $i$ choosing start date $s_i$ and end date
  $e_i$ for their beam reservation.

  As manager, most days you work from an office an hour drive away from the
  main facility.  However, you like to go by the facility to meet each research
  group for at least a few minutes to hear about their project and how it's
  progressing.  If you do this, you spend the whole day at the main facility
  instead of your office, which is more than enough time to meet all the groups
  present that day.

  Given a fixed schedule for the start and end dates of each group's visit,
  your goal is to \emph{minimize the number of days} you need to be present at
  the beam in order to meet all of the research groups.  (That is, you want to
  find the smallest set of possible days to visit the facility such that you
  will overlap with each of the $n$ groups working at the facility.)

  \begin{subproblems}
    \item Describe a greedy algorithm that, given the start dates $s_i$ and end
      dates $e_i$ for each of the $n$ groups scheduled, determines a smallest
      set of dates in the range 1 to $D$ such that you can meet all the groups.
      Don't worry about giving big-O running time.  \emph{Your algorithm
        description requires just  a few sentences of clear English.  That's
        sufficient!  If you wish to also include some high-level pseudo-code,
      that's fine, but not necessary.}
    \item Prove the correctness of your algorithm using mathematical induction
      and the safe choice method.
  \end{subproblems}
\end{problem}

\begin{solution}
\end{solution}

\begin{problem}[Focus Problem: Making Change Revisited, 50]%
  The Republic of Shmorbodia\footnote{The Republic of Shmorbodia is an official
  sponsor of this course.} uses coin denominations that are Fibonacci numbers.
  Recall that the Fibonacci numbers are defined as follows:  $F_0 = 0$, $F_1 =
  1$, and $F_n = F_{n-1} + F_{n-2}$ for $n ≥ 2$.  The Shmorbodian currency
  system comprises coins $F_2, \dotsc, F_k$ for some fixed constant $k$.  (Note
  that we start with $F_2$ rather than $F_1$ since $F_1 = F_2 = 1$.)  That is,
  the coin values are consecutive Fibonacci numbers up to some Fibonacci number
  $F_k$.  In this problem, you'll show that the greedy algorithm for making
  change (i.e., use one coin of the largest denomination possible, reduce the
  balance by that amount, and repeat until the balance is zero), is optimal in
  the sense that it uses the least possible number of coins.

  As usual in proving the optimality of greedy algorithms, you should use the
  safe choice method.  Recall that when we proved the optimality of the
  algorithm for a ``geometric'' denomination system $b^0, b^1, \dots, b^k$ for
  integers $b > 1$ and $k ≥ 0$, we needed to make some observations about what
  an optimal solution could look like (i.e., an optimal solution has at most
  $b-1$ coins of type $b^{\ell}$ for any $\ell < k$) and we derived an identity
  that allowed us to show that the greedy choice is part of an optimal solution
  ($b^0 + b^1 + \dots + b^i = \frac{b^{i+1}-1}{b-1}$).  You'll use an analogous
  approach here.

  First, recall that in a geometric denomination system, we showed that the
  greedy choice is part of \emph{every} optimal solution.  That is, we
  \emph{must} make that greedy choice to get an optimal solution.  That's a
  stronger result than we actually needed.  It would suffice to just argue that
  \emph{there exists} some optimal solution that uses the greedy choice.  In a
  Fibonacci denomination system, there can exist multiple equally optimal
  solutions, including some that do not make the greedy choice.  For example,
  consider the Fibonacci system with coin values of $1, 2, 3$.  If we wish to
  make $4$ cents of change, we can do that with two coins as $3$ and $1$ (this
  is the result of making the greedy choice) or, alternatively, as $2$ and $2$.
  So, note that we are \emph{not} claiming here that the greedy algorithm is
  the only way to get an optimal solution.  Rather, we're claiming that the
  greedy algorithm is guaranteed to give an optimal solution.

  \begin{subproblems}
    \item First, you'll need to make some observations about what an optimal
      solution looks like.  Assume that our coins are $F_2, F_3, \dots, F_k$.
      \begin{subproblems}

        \item Prove that no optimal solution uses two consecutive coin types
          $F_{i}$ and $F_{i+1}$ for $i+1 < k$.

        \item Prove that $2F_{i} = F_{i+1} + F_{i-2}$, for $i ≥ 2$.  (No need
          for induction here, you can just use the definition of Fibonacci
          numbers and a tiny bit of algebra.)

        \item \label{part:safe} Prove that there exists an optimal solution
          that does not use two or more coins with value $F_i$ where $i < k$.
          In other words, although we may need to use multiple coins of highest
          denomination, $F_k$, there exists an optimal solution that uses at
          most one coin of each lower denomination.  (An extremality argument
          will be useful here.  In particular, you may wish to consider the set
          of optimal solutions that use the largest number of $F_k$ coins.  In
          that set of optimal solutions, consider the subset that uses the
          largest number of $F_{k-1}$ coins.  Now, consider the subset of that
          subset that uses the largest number of $F_{k-2}$ coins, etc.  In this
          way, you'll choose not \emph{any old} optimal solution but rather a
          particular ``extremal'' one.

          That will allow you to show that there must exist \emph{some} optimal
          solution that does not use two or more coins with value $F_i$ where $i <
          k$.)
      \end{subproblems}

    \item \label{part:ident} Next, we'll need some identities that we'll use in
      our proof.
      \begin{subproblems}
        \item Prove by induction (or strong induction) that for all $i ≥ 0$,
          $F_0 + F_2 + F_4 + \dots + F_{2i} = F_{2i+1} - 1$ (cite this as
          ``Identity 1'')
        \item Prove by induction (or strong induction) that for all $i ≥ 1$,
          $F_1+ F_3 + F_5 + \dots + F_{2i-1} = F_{2i}$ (cite this as ``Identity
          2'')
      \end{subproblems}
    \item Finally, use the safe choice method  to prove that the greedy
      algorithm always gives an optimal solution for a Fibonacci denomination
      system with coin values $F_2, F_3, \dots, F_k$ for any fixed constant $k
      ≥ 2$.  (Remember that this is a proof by strong induction.  Just as in
      the proof that we used for the change problem in class, this will be
      strong induction on the amount of money $n$ for which we want to make
      change.  The result from \cref{part:safe} and the identities in
      \cref{part:ident} will be useful.)
  \end{subproblems}

\end{problem}

\begin{solution}
\end{solution}

\begin{problem}[Tree Stumpers, 30]

  After learning about Prim's and Kruskal's algorithms, your friends have been
  pining for some alternative algorithms for finding minimum spanning trees, so
  they've gone out on a limb and proposed some new algorithms.  (No more tree puns
  from this point onwards!)  After unsuccessfully leafing through algorithms
  textbooks to determine if their algorithms are correct, they are stumped!  So,
  they've decided to branch out and enlist your help determining which of these
  algorithms work.

  For each of the two algorithms below, either prove the correctness of the
  algorithm, or give a specific (small) graph as a counterexample.  If you give
  a counterexample, make sure to explain why it is a counterexample, i.e., how
  the algorithm's tree would differ from the optimal tree.  \textbf{You should
    assume that the graph $G = (V, E, w_e)$ has distinct edge weights $w_e$ for
  each edge $e \in E$.}

  \begin{subproblems}
    \item One friend, May B.~So, suggests the following recursive algorithm
      \(\Ident{MayBTree}(G)\): If $G$ only has one vertex, return an empty
      tree. Otherwise, arbitrarily split the graph in half, creating two
      subgraphs $Y$ and $Z$ of $G$ that each contains roughly half the vertices
      in $V$. Recursively call this function for each graph to get one tree for
      each, $S = \Ident{MayBTree}(Y)$ and $T = \Ident{MayBTree}(Z)$.  Then,
      find the lowest-cost edge $e$ in the original graph such that one
      endpoint is in $Y$ and the other is in $Z$.  Return $S + T + \{e\}$.

    \item Another friend, Quincy Vablie, suggests the following algorithm: Like
      Kruskal's algorithm, each vertex starts in its own component.  Start with
      an empty set of edges $T = \{\}$.  The algorithm proceeds in rounds: At
      each round, each connected component (in an arbitrary order) identifies
      the edge of least weight that has one endpoint in that component and one
      endpoint in a different component.  (We call this a ``least weight
      outgoing edge,'' and note that two components may have the same least
      weight outgoing edge!) Once each component has identified its least
      weight outgoing edge, all of those edges are added to $T$, resulting in a
      new set of connected components.  This process is repeated in rounds
      until there is only one connected component remaining.  At that point,
      the set $T$ is returned.
  \end{subproblems}

\end{problem}

\begin{solution}
\end{solution}

\begin{problem}[Optional Bonus Problem: The Arithmetic and Geometric Mean, 20]

  Let $x_{1}, \dotsc, x_{n}$ be positive real numbers.  The \emph{arithmetic
    mean} of these numbers is defined to be $\frac{x_{1} + x_{2} + \dotsb +
  x_{n}}{\ n}$ and the \emph{geometric mean} is defined to be $(x_{1}x_{2}
  \dotsm x_{n})^{1/n}$.  In this problem we show that the arithmetic mean of
  $n$ numbers is at least as large as the geometric mean of those numbers.
  \begin{subproblems}
    \item Use induction to show that if $x_{1} x_{2} \dotsm x_{n} = 1$ then
      $x_{1} + x_{2} + \dotsb + x_{n} ≥ n$.  (Beware of the induction pitfall
      mentioned in the handout on writing inductive proofs.)

    \item Use this fact to show that the arithmetic mean is at least
      as large as the geometric mean.  (No induction required here; just
      a little algebra.)
  \end{subproblems}

\end{problem}

\begin{solution}
\end{solution}

\end{document}
