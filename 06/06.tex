\documentclass{ks-pset}

\usepackage{ks-cs}

\title{Homework 6: Amortization and Graphs!}

\begin{document}

\begin{itemize}

  \item Please read \emph{Handout: Amortized Analysis} before embarking on this
    assignment.

  \item If a problem asks you to use the accounting method, you need to first
    list the number of rubles charged for each operation. Next, describe how
    these rubles will be used in each operation, including where rubles will be
    stored on or taken from the data structure. (Remember that every constant
    set of work needs at least one ruble to pay for it!) Then, prove that this
    rubling scheme will work to pay for work by giving an invariant that
    ensures enough rubles will exist in the data structure at any time to help
    pay for all work, then showing that if it holds before each operation, it
    will still hold after. Wrap up by computing the new upper bound of the
    running time of \(n\) operations as \(n\) times the highest ruble charge
    for an operation.

  \item If a problem asks you to use the potential method, you should first
    state the potential function \(Φ\) you intend to use. (It may be helpful to
    name quantities from the data structure that you intend to use in your
    potential function.) Then, prove your potential function is valid, i.e.,
    that it starts out at zero and will always remain non-negative. Finally,
    describe the new amortized cost of each operation by summing the original
    operation running time and \(ΔΦ\) for that operation. Wrap up by computing
    the new upper bound of the running time of \(n\) operations as \(n\) times
    the highest new amortized cost of an operation.

  \item In either argument, it may make sense to split into different ``cases''
    for what a particular operation does (as we did in class with extendible
    arrays, where only sometimes a copy operation would need to happen). This
    is okay, as long as you make sure to consider each possible case.

\end{itemize}

\newpage

\begin{problem}[Minqueues!, 35]

  A Minqueue is an abstract data type (sometimes called an ADT) that supports
  the following operations:

  \begin{description}[nosep]
    \item[\(\Ident{enqueue}(x)\)] Inserts the number \(x\) into the Minqueue.
    \item[\(\Ident{dequeue}()\)] Removes the element that has been in the
      Minqueue for the longest time.
    \item[\(\Ident{find_min}()\)] Returns the smallest value in the Minqueue
      but \emph{does NOT} remove it from the Minqueue.
  \end{description}

  Your officemate at Millisoft, Dr.~Anna Litik, has proposed a clever data
  structure for this problem which she calls ``Real Queue and Helper Queue.''
  Here's how it works:  When an element is enqueued, it is placed into a
  regular queue (implemented as a doubly linked list).  However, it is
  \emph{also} placed at the tail of a special ``helper queue'' (also
  implemented as doubly linked list).  The helper queue will always contain a
  subset of those elements in the real queue in sorted order from small
  elements at the front to large elements in the back.  In particular, when an
  element \(x\) is inserted at the tail end of the helper queue, it checks to
  see if it is smaller than the element right in front of it in the helper
  queue.  If so, it removes the element in front of it in the helper queue and
  again compares itself to its predecessor.  It repeatedly removes its
  predecessors until it gets to a point that its predecessor is smaller than
  it!  To dequeue, we simply remove that element from the head of the real
  queue.  We also check to see if it is at the head of the helper queue, in
  which case we remove it from that queue as well.  Finally, to determine which
  element is the minimum, we simply look at the front of the helper queue!

  \begin{subproblems}
    \item Use the accounting method to prove that the total running time of
      \emph{any} sequence of \(n\) operations is \(\AsymptoticO(n)\).
    \item Show the same amortized bound of \(\AsymptoticO(n)\) for \(n\)
      operations with the potential method.
  \end{subproblems}

  \begin{hint}
    Be careful to account for all of the work performed by this data structure.
    Every constant amount of work, including comparisons, should be able to
    identify a ruble that will be used to pay for that work.
  \end{hint}

\end{problem}

\begin{solution}

  \begin{subproblems}
    
    \item

    \item 

  \end{subproblems}

  \paragraph{Self-assessment} Self assessment goes here.
\end{solution}


\begin{problem}[\Py{PrimDijkstra}, 65]

  In this problem, you'll implement Prim's MST algorithm and Dijkstra's
  shortest path algorithm in one function called \Py{PrimDijkstra}. Please
  implement this code using \textbf{Python 3}. (Python 2 implementations may
  not pass our testing script.) Begin by downloading \Ident{Heap.py} and
  \Ident{PrimDijkstra.py}, included in the problem instructions online.

  \begin{description}

    \item[\Ident{Heap.py}] \Ident{Heap.py} is the starter code for the
      \Py{Heap} class.

      \begin{itemize}

        \item Note that the heap stores \Ident{Item} objects. An \Ident{Item}
          comprises a \textbf{value}, a \textbf{vertex} number, and a
          \textbf{predecessor}:

          \begin{itemize}

            \item The \textbf{value} is the number associated with a vertex.
              For example, in Dijkstra's Algorithm, it's the current distance
              to that vertex. Recall that the start vertex initially has a
              value of \(0\) and all other vertices have values of infinity.

            \item The \textbf{vertex} is just the name of the vertex. The
              vertex number is the name of that vertex.  \emph{We use a
                convention in which the \(n\) vertices are numbered \(0\)
              through \(n-1\)}.

            \item Finally, the \textbf{predecessor} is the ``breadcrumb'' that
              ultimately lets us reconstruct a shortest path from this vertex
              back to the starting vertex. Specifically, if vertex \(42\) got
              its best offer from vertex \(57\), then vertex \(42\)'s
              predecessor would be \(57\). In other words, when a vertex \(v\)
              makes a relaxation offer (aka decreaseKey) to vertex \(w\) and
              vertex \(w\) accepts that offer, vertex \(w\) records vertex
              \(v\) as its predecessor.

          \end{itemize}

        \item The Heap comprises two lists, one called \Py{self.array} and the
          other called \Py{self.index}.

          \begin{itemize}

            \item The list \Py{self.array} is the array implementation of the
              heap. It contains \Py{Items}. The \Py{Item} with the smallest
              value is at the top of the heap, and thus at index 1 in
              \Py{self.array}.  \emph{It's important that we don't use index
              0!} The reason is that the array implementation of the heap tree
              only works if the root is at index 1. This allows us to use the
              rule that the left and right children of the node at index \(i\)
              are at indices \(2i\) and \(2i+1\), respectively, in the array
              implementation of the heap. Similarly, the parent of the node at
              index \(i\) in the heap is at index \(i/2\) rounded down. This
              doesn't work if we used index \(0\) as the first element in the
              heap.

            \item The list \Py{self.index} allows us to find where a given
              vertex is stored in the heap represented by \Py{self.array}. For
              example, if vertex \(4\) is currently stored at index \(17\) in
              \Py{self.array}, then \Py{self.index[4]} would be \(17\). Refer
              to the lecture notes and video on Dijkstra's Algorithm to remind
              yourself of how this array works.

          \end{itemize}

        \item \emph{Test your Heap!} We recommend writing some of your own very
          small tests to test each method that you write. Test that method
          before writing the code for the next method. Designing your own small
          tests is a very important skill as a software developer. At the end,
          you should also use our provided \Ident{Heaptest.py} code. Run it by
          typing \mintinline{bash}{python Heaptest.py} in the shell.

      \end{itemize}

    \item[\Ident{PrimDijkstra.py}] \Ident{PrimDijkstra.py} is the main program
      and it uses the Heap. The functions in this file are described in detail
      below.

      \begin{description}
        \item[\Ident{PrimDijkstra(graph, start, alg)}] The main function in
          \Ident{PrimDijkstra.py} is called
          \[
            \Py{PrimDijkstra(graph, start, alg)}.
          \]
          It takes an adjacency list, a start vertex, and an algorithm (either
          ``P'' for Prim or ``D'' for Dijkstra) and returns an adjacency list
          of either the minimum spanning tree or shortest path tree,
          respectively.

          An example of an input adjacency list is given below (and is also
          given in the \Ident{PrimDijkstra.py} file). The adjacency list is a
          list of lists, where the item at index i is a list of ordered pairs
          of the form (neighbor, weight) where neighbor is the neighbor vertex
          and weight is the weight of the edge from vertex i to vertex
          neighbor. Recall that \emph{in our convention, the first vertex is
          vertex \(0\).}  (Don't confuse this with the heap, where the first
          index in the array is \(1\).  Those are two entirely different
          things.  Vertices are elements of the graph.  The heap array is just
          a representation of the binary tree heap and it starts storing items
          at index \(1\).)

          \inputminted{python}{template/02-graph1.py}

          \tikzset{
            vert/.style={
              circle,
              minimum size=3pt,
              draw,
              inner sep=1pt,
              font=\small,
            },
            ex/.pic={
              \begin{scope}[x=4em, y=4em]

                \draw
                (0,0) node[vert](6){\(6\)}
                ++(1,0) node[vert](5){\(5\)}
                ++(30:1) node[vert](4){\(4\)}
                ++(-30:-1) node[vert](3){\(3\)}
                ++(-1,0) node[vert](2){\(2\)}
                ++(30:-1) node[vert](1){\(1\)}
                ($ (1)+(0,1) $) node[vert](0){\(0\)}

                (6) -- node[midway, below]{\(80\)} (5)
                -- node[midway, below, sloped]{\(60\)} (4)
                -- node[midway, above, sloped]{\(90\)} (3)
                -- node[midway, above]{\(70\)} (2)
                -- node[midway, above, sloped]{\(20\)} (1)
                -- node[midway, below, sloped]{\(49\)} (6)
                -- node[midway, left]{\(30\)} (2)
                -- node[midway, above, sloped]{\(10\)} (5)
                -- node[midway, right]{\(40\)} (3)

                (1)
                -- node[midway, left]{\(10\)} (0)
                -- node[midway, above, sloped]{\(100\)} (2);

              \end{scope}
            },
          }

          Figure~\ref{figure:graph} shows this graph in pictorial form.

          \begin{aside}
            \begin{center}
              \tikz{\pic{ex}}
              \captionof{figure}{Example graph.}
              \label{figure:graph}
            \end{center}
          \end{aside}

          In Figure~\ref{figure:results} (left) is the minimum spanning tree in
          this graph found using Prim's Algorithm starting at vertex~\(0\).  On the
          right is the shortest path tree found using Dijkstra's Algorithm
          starting at vertex~\(0\).

          \begin{aside}
            \begin{center}
              \begin{tikzpicture}
                \pic{ex};
                \draw[ultra thick] (0)--(1)--(2)--(5)--(3) (5)--(4) (2)--(6);
              \end{tikzpicture}
              \qquad
              \begin{tikzpicture}
                \pic{ex};
                \draw[ultra thick] (0)--(1)--(2)--(5)--(3) (5)--(4) (1)--(6);
              \end{tikzpicture}

              \captionof{figure}{Results of running Prim's (left) vs.
              Dijkstra's (right) algorithm on the graph.}
              \label{figure:results}
            \end{center}
          \end{aside}

          Here's what happens when we run \Py{PrimDijkstra} on this graph…

          \inputminted{python}{template/02-pd1-p.py}

          In this case, we run Prim's Algorithm (that's the ``P'') in the third
          argument and we get the MST shown on the left side of the figure. The
          output is an adjacency list of the MST rooted at vertex \(0\) and
          \emph{directed towards the leaves}.  Although the graph is not
          actually directed, we treat this adjacency list as being directional
          because vertex \(0\) invited vertex \(1\) in the MST. That is, Prim
          added an edge from vertex \(0\) ``towards'' vertex \(1\).  Vertex 1
          invited vertex \(2\), but it did not invite vertex \(0\).

          So, vertex \(0\) has one edge towards vertex \(1\). Vertex \(1\) has
          one edge towards vertex \(2\). Vertex \(2\) has two edges towards
          vertex \(5\) and towards vertex \(6\). Vertex \(3\) has no edges
          towards any other vertex. Vertex \(4\) has no edges towards any other
          vertex. Vertex \(5\) has edges towards vertices \(3\) and \(4\) and
          vertex \(6\) has no edge towards any other vertex.

          In the second case, we see the shortest path tree found by running
          Dijkstra's algorithm from vertex \(0\) as follows, where the ``D'' in
          the third argument stands for Dijkstra.

          \inputminted{python}{template/02-pd1-d.py}
        
          Again, vertex \(0\) invited vertex \(1\). Vertex \(1\) invited
          vertices \(2\) and \(6\), etc.

          \begin{note}
            Your \Py{PrimDijkstra} function will be around 25 lines long. As
            discussed in class, the only distinction between Prim and Dijkstra
            should be something like this (you can use this code verbatim if
            you wish).  The important thing here is that you should not have
            separate logic for Prim and Dijkstra beyond one such conditional
            statement that dictates how an offer is made to a neighboring
            vertex.

            \inputminted{python}{template/02-pd-difference.py}
          \end{note}

        \item[\Py{navigate(graph, start, end)}] Finally, your
          \Ident{PrimDijkstra.py} file should have a function called
          \[
            \Py{navigate(graph, start, end)}
          \]
          that gives a shortest path from the start vertex to the end vertex in
          the given graph. Here are a few examples of this function in action:

          \inputminted{python}{template/02-navigate.py}

          The \Py{navigate} function will first call \Py{PrimDijkstra} to
          compute a shortest path tree from the start vertex to all reachable
          vertices in the graph.  It will then search that tree for a path from
          the start vertex to the desired end vertex.  You may wish to
          introduce one or more helper functions to help with that, but the
          \emph{total} amount of code in navigate and its helper function(s)
          shouldn't be more than around 10--15 lines total.  Moreover,
          \Py{navigate} doesn't need to be as fast as possible, as long as it's
          not asymptotically slower than the \Py{PrimDijkstra} function itself!
          In other words, since \Py{PrimDijkstra} runs in time $\AsymptoticO(m
          \log n)$, navigate may be implemented any way that you like as long
          as it doesn't exceed $\AsymptoticO(m \log n)$. (The reason for this
          requirement is to make sure that \Ident{PrimDijkstra}'s running time
          is the ``dominant'' term in the running time.)

      \end{description}
  \end{description}

  Please submit all of your code and your PDF in a single upload. It should
  contain \Ident{Heap.py}, \Ident{PrimDijkstra.py}, and \emph{your PDF solution
  for Problem 1 (Minqueues)}.

\end{problem}

\begin{solution}

  \paragraph{Self-assessment}
\end{solution}

\end{document}
