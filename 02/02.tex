\documentclass{ks-pset}

\usepackage{ks-cs}

\title{Homework 2}
\author{}
\date{2022 February 9 (Wednesday)}

\begin{document}

\begin{itemize}
  \item Please read \emph{Handout 2:  Dynamic Programming} before embarking on
    this assignment.  It's a good guide of how to write a complete dynamic
    programming proof!
  \item At the end of each problem, please include a brief self-assessment. If
    you know that your solution is not quite complete or correct, let us know
    that you recognize that and where you believe that the issue arises. If
    you're confident in your solution, let us know that as well! Generally,
    just one sentence suffices here. Part of your score will be based on an
    accurate self-assessment of your work.
  \item Make sure to list all of your collaborators on this assignment at the
    top.
\end{itemize}

\newpage

\begin{problem}[Sho'nuffian Palindromes! -- 40]

  The Sho'nuffian Security Agency embeds secret messages as palindrome
  subsequences of longer strings.  For example, the string

  \mintinline{text}{babbzmqanatplanxzatcanazlxpanamxazz}

  contains inside it the palindrome \mintinline{text}{amanaplanacanalpanama} as
  seen below:

  \inputminted{text}{palindrome.txt}

  Your job is to describe and analyze an efficient dynamic programming
  algorithm for finding the longest palindromic subsequence in a given string
  \(S\) of length \(n\). To that end, follow the steps outlined in Handout 2, which
  begins by describing a recursive solution and culminates in the DP solution
  and its analysis, with various intermediate steps along the way. At first,
  just describe how to find the length of the longest palindrome and, at the
  very end, describe the extra part using ``breadcrumbs'' to reconstruct an
  actual solution. Make sure to include the proof of correctness of the
  recursive algorithm and the big-O running time!

  \begin{note}
    Like in class, when you index into a string, we'll ask you to write out
    relevant indices within the larger string, e.g. \(S[1 \dotso i]\) could
    represent the first \(i\) characters of a string, or 0 characters if \(i =
    0\). You should be specific and consistent with these indices in your
    recursive function and your DP table; in addition, these indices may help
    you express what you're inducting over for your proof!
  \end{note}

  \begin{note}
    A tempting approach to this problem is to find the longest common
    subsequence (LCS) of the string and its reversal.  This doesn't always
    work.  For example, given the string ``ABCAB," one longest common
    subsequence of the string and its reversal is the string ``ACB'' which is
    not a palindrome.  You should, therefore, describe a new recursive
    algorithm that does not use LCS.
  \end{note}

\end{problem}

\begin{solution}
  \paragraph{Self-assessment}
\end{solution}

\begin{problem}[Smart Lumber -- 40]

  At the Unseen University (a well-known wizarding university on the
  Discworld), several keen explorer-wizards recently reported back with an
  astounding discovery: they have located a large forest of \emph{sapient
  pearwood}, an astonishingly rare material of immense value for constructing
  high-quality sentient furniture. Since the discovery, the wizards have begun
  their own enterprise selling ``smart lumber'' called Pearwood Industries
  (PI). Being somewhat new to the lumber business, and knowing your algorithmic
  wizardry, they come to you with a problem.

  Given an \(m×n\) cross-section of lumber (\(m\) and \(n\) are integers and
  the units are in centimeters), PI would like to cut the piece of lumber into
  some number of smaller pieces for sale.  Each section will be a rectangle
  with integer dimensions.  The value for any \(k×ℓ\) piece can be looked up in
  constant time, though the value is not necessarily related to the area of the
  piece, as some dimensions are simply in higher demand.  For example, a
  \(3×4\) piece might be worth \$42 (in Ankh-Morporkian dollars) while a
  \(6×6\) piece might be worth \$8.   

  Your wizarding friends have constructed a special saw that is capable of
  making a cut -- horizontally or vertically -- across any rectangular section
  of sapient pearwood.  (Note that a cut is made across the entire section of
  the given piece; no partial cuts are possible.) Once you have fully cut a
  piece of wood, you may then cut each of those two resulting pieces into
  further pieces independently.

  \textbf{Describe and analyze an efficient dynamic programming algorithm for
    determining how to subdivide a given \(m×n\) piece to maximize the total
  value of the resulting pieces.}  Use the process and writing style from
  Handout 2.  For simplicity of exposition, you should describe an algorithm
  that returns a number which is the maximum possible revenue.   No proof of
  correctness is required, but you should bound the big-O running time. At the
  very end,  in just a few additional sentences, explain how you would modify
  your algorithm to return the actual sequence of cuts that result in the
  pieces that make up this optimal revenue. \emph{You don't need to analyze the
  big-O running time of reconstructing the cut sequence.}

\end{problem}

\begin{solution}
  \paragraph{Self-assessment}
\end{solution}

\begin{problem}[The Arithmetic and Geometric Mean -- 20]

  Let \(x_{1}, \dotsc, x_{n}\) be positive real numbers.  The \emph{arithmetic
  mean} of these numbers is defined to be \(\frac{x_1+x_2+\dotsb+x_n}{n}\) and
  the \emph{geometric mean} is defined to be \((x_1 x_2 \dotsm x_n)^{1/n}\). In
  this problem we show that the arithmetic mean of \(n\) numbers is at least as
  large as the geometric mean of those numbers.
  \begin{subproblems}
    \item Use induction to show that if \(x_1 x_2 \dotsm x_n=1\) then
      \(x_1+x_2+\dotsb+x_n≥n\).  (Beware of the induction pitfall mentioned in
      the handout on writing inductive proofs.)
    \item Use this fact to show that the arithmetic mean is at least as large
      as the geometric mean.  (No induction required here; just a little
      algebra.)
  \end{subproblems}

\end{problem}

\begin{solution}
  \paragraph{Self-assessment}
\end{solution}

\end{document}
