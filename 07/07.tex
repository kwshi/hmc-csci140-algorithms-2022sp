\documentclass{ks-pset}

\usepackage{ks-cs}

\title{Homework 7: Graphs and Flow!}
\author{}
\date{2022 April 13 (Wednesday)}

\begin{document}

\begin{problem}[SpaceY, 50]

  SpaceY is a company that builds rockets and sends payloads into space for
  other companies.  On a given mission, SpaceY will consider a set of
  experiments that clients would like to conduct (e.g., ``Does zero-gravity
  compromise the running time of algorithms?'').  In general, SpaceY will
  receive more requests from clients than it can fulfill, so its goal is to
  fulfill those requests that result in the most profit.

  Let \(E=\Set{E₁,E₂,\dotsc,Eₘ}\) denote the set of experiments under
  consideration.  Let \(pⱼ\) denote the amount of money the prospective client
  will pay to conduct experiment \(Eⱼ\). Let \(I=\Set{I₁,I₂,\dotsc,Iₙ}\) denote
  a set of instruments. For each experiment \(Eⱼ\), let \(Rⱼ\) be the subset of
  \(I\) that contains all of the instruments needed to conduct experiment
  \(Eⱼ\).  Notice that this allows a single instrument to be used in multiple
  experiments. The instruments are potentially large and heavy and the cost of
  taking instrument \(Iₖ\) is \(cₖ\) dollars.  Your job is to determine which
  experiments should be performed in order to maximize the net revenue, which
  is the total income from the performed experiments minus the total cost of
  all instruments carried.  Amazingly, this problem can be solved using
  \textbf{network flow}!

  We'll construct a network flow problem as follows: The network contains a
  source vertex, \(s\), vertices \(I₁,I₂,\dotsc,Iₙ\), vertices
  \(E₁,E₂,\dotsc,Eₘ\), and a sink vertex \(t\). For each instrument \(Iₖ\)
  there is a directed edge from \(s\) to vertex \(Iₖ\) with capacity \(cₖ\)
  (the cost of taking this instrument).  For each experiment \(Eⱼ\) there is an
  edge from vertex \(Eⱼ\) to \(t\) with capacity \(pⱼ\) (the payment for this
  experiment). Finally, if instrument \(Iₖ\) is in set \(Rⱼ\) (meaning that
  instrument \(Iₖ\) is needed for experiment \(Eⱼ\)) then there is a directed
  edge from vertex \(Iₖ\) to vertex \(Eⱼ\) with \textbf{infinite} capacity.

  \begin{subproblems}

    \item First, try this.  Consider a situation in which there are three
      experiments \(E₁, E₂, E₃\) which will bring in \(10\), \(6\), and \(6\)
      dollars, respectively.  Also there are four instruments \(I₁\), \(I₂\),
      \(I₃\), and \(I₄\) which cost \(3\), \(2\), \(5\), and \(7\) dollars,
      respectively to take on the shuttle.  Experiment \(E₁\) requires
      instruments \(I₁\) and \(I₂\), experiment \(E₂\) requires instruments
      \(I₁\) and \(I₃\), and experiment \(E₃\) requires instruments \(I₃\) and
      \(I₄\).  Using brute-force, enumerate all seven possible combinations of
      experiments that could be taken and determine which combination is the
      most profitable.  What is this combination and what is the net revenue?

    \item For the example problem above, construct the corresponding network
      flow problem.  Find the maximum flow in the network (show a picture of
      the residual graphs at each step) and then find the corresponding cut
      whose capacity is equal to this flow.  (Remember, this cut is found from
      the last residual network when running the Ford-Fulkerson algorithm!)

    \item Let's let \(S\) denote the vertices that are on the same side of the
      above cut as vertex \(s\) and let \(T\) denote the vertices that are on
      the same side of the cut as \(t\).  What do you notice about the
      instruments and experiments that are in the set \(T\)?

    \item Now, let \(τ\) be the sum of the payments that SpaceY would receive
      for taking all of the possible experiments.  In this case,
      \(τ=10+6+6=22\). From \(τ\), subtract the capacity of the cut you found
      above.  Surprise!  What is this number and how does it appear to relate
      to this problem?

    \item Finally, we're ready to generalize all of this into an efficient
      algorithm for solving the profit maximization problem in general.  Assume
      that we've set up a network flow problem corresponding to a given set of
      experiments and instruments. Show that for any cut with finite total
      capacity, if an experiment \(Eⱼ\) is in \(T\) (the side of the cut
      containing vertex \(t\)), then all of the instruments used in this
      experiment must \textbf{also} be in \(T\).

    \item Now argue that the maximum net revenue that can be achieved is simply
      the total sum \(τ\) of the payments that would be received for taking all
      of the experiments minus the capacity of the minimum cut.

    \item Now just summarize all of this by describing the algorithm,
      step-by-step, for finding the maximum net revenue, given a set of
      experiments, instruments, and the corresponding payments and costs.  Give
      a careful derivation of the worst-case running time of the algorithm
      assuming there are \(m\) experiments and \(n\) instruments. The running
      time may also depend on the instrument costs and experiment payment
      values.

  \end{subproblems}

\end{problem}


\begin{solution}

  \paragraph{Self-assessment}
\end{solution}


\begin{problem}[Hurts' On-board Navigation System, 50]

  This problem, brought to you by Hurts Car Rental, is a good example of how
  \emph{using existing algorithms} can be the secret to all happiness!

  Hurts Car Rental has designed a new generation of alternative fuel vehicles.
  The new vehicles use a special fuel comprising a finely minced mixture of
  algorithms slides, chocolate fudge Pop Tarts, Spam, late 90's hip-hop CDs,
  and boba milk tea. Due to this rather unusual fuel requirement, there are
  only certain cities in the country where the vehicles can be refueled. Thus,
  to get from the start city to the destination city, the driver must plan a
  route that ensures that the car can be refueled along the way.

  The on-board computer on such a vehicle contains a weighted directed graph in
  which the \(n\) vertices represent cities, the \(m\) directed edges represent
  one-way roads, and the weights on the edges represent the (positive integer)
  lengths of the roads. The graph, of course, can have cycles! However,
  \textbf{you should assume that there is some constant \(C\) such that every
    vertex has in-degree and out-degree (number of entering edges and number of
  exiting edges, respectively) at most \(C\).}

  The computer knows which select cities have filling stations. To use the
  navigation system, the user will enter the starting city, the destination
  city, and the range of the vehicle on a full tank. (You may assume that the
  starting city and the destination cities, being Hurts rental cities, both
  have filling stations - and that the car starts with a full tank of fuel.)
  The computer will respond with the \emph{shortest} route from the starting
  city to the destination city that ensures that the vehicle doesn't run out of
  fuel, or will determine that no such route exists.

  \begin{subproblems}

    \item Give an algorithm for solving this problem that uses existing graph
      algorithms.  Be sure not to change existing algorithms---use them
      ``off-the-shelf'' so that you can leverage their known proofs of
      correctness and running times.  Choose the existing algorithms that allow
      you to solve this problem with the best running time and derive the
      running time of your solution.

      \begin{note}
        You have several shortest path algorithms at your disposal: Dijkstra,
        Bellman--Ford, Floyd--Warshall, and Johnson's Algorithm. Some will be
        faster than others for this problem, so you'll want to choose carefully
        for full credit.
      \end{note}

    \item Prove that your algorithm finds an optimal solution.

      \begin{note}
        Part of the benefit of using existing algorithms in your solution is
        that you can use their correctness to help prove the correctness of the
        full algorithm faster. However, for this to work, you must not tamper
        with the original algorithms, or you would have to re-prove their
        correctness!
      \end{note}

  \end{subproblems}

\end{problem}

\begin{solution}

  \paragraph{Self-assessment}
\end{solution}

\end{document}
