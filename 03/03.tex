\documentclass{ks-pset}

\usepackage{ks-cs}

\title{Homework 3: Dynamic Programming II}
\author{}
\date{2022 February 16 (Wednesday)}

\begin{document}

\begin{problem}[Making Change Revisited, 40]

  In class and on Homework 1, we explored the problem of making change using
  the minimum number of coins.  We saw two types of denomination systems
  (geometric and Fibonacci) where the greedy algorithm is guaranteed to produce
  an optimal solution.  But, the greedy algorithm doesn't produce optimal
  solutions for all denomination systems.  For example, if the denomination
  system in Sho'nuff has coins of values \(1, 7, 8\) and we wish to make change
  for \(14\), the greedy algorithm will use one coin of value \(8\) and six
  coins of value \(1\), for total of seven coins.  However, giving two coins of
  value \(7\) is optimal.

  In this problem, you'll design a dynamic programming algorithm for the
  general case of any set of coin denominations, assuming that a coin of value
  \(1\) (a ``penny'') is always part of the denomination system.  Assume that
  you're given a list of coin types \(C\) sorted from smallest to largest
  value, so that \(C[0] = 1\).  Let \(n\) denote the number of coin types in
  the system, so that the largest index in the coin array is \(n-1\).  You are
  also given a non-negative integer \Ident{amount} indicating the amount of
  money that you need to make up from those coins.

  \begin{itemize}

    \item First, give pseudo-code (you can use your own conventions for
      pseudo-code as long as it's clear) for solving the problem recursively.
      We've provided a verbatim environment below for you to fill in the
      pseudo-code.  Be sure to put the input arguments in the signature and
      explain what they mean in the docstring.  For the recursive function,
      please expose the indices in the signature as discussed in class on
      Thursday.  For example, for the RNA folding problem, we wrote a
      pseudo-code signature \(\Ident{fold}(i, j)\) or
      \(\Ident{fold}(\Ident{RNA}[i\dotso j])\) and wrote in the docstring
      \texttt{Returns the optimal fold score for the substring of the RNA from
      index \(i\) to index \(j\) inclusive}.  This helps the reader understand
      your plan and will ultimately help you construct the corresponding DP.
      For the recursive change function, the arguments might be \(i\) and
      \Ident{amount} and the docstring would read \texttt{Returns the minimum
        number of coins required to make amount using only coins from \(C[1]\)
      through \(C[i]\) inclusive.} Or, the function might take two index
      arguments \(i\) and \(j\) and the docstring would read \texttt{Returns
        the minimum number of coins required to make amount using only coins
      from \(C[i]\) through \(C[j]\) inclusive.}  In general, fewer arguments
      will result in a lower-dimensional DP table which saves memory. So,
      when possible, try to keep the number of arguments small.

      The last argument will be \Ident{amount}.

      \begin{aside}
        \inputminted{python}{template/recursive_change.py}
      \end{aside}

    \item Next, give pseudo-code for the dynamic programming solution.

      \begin{aside}
        \inputminted{python}{template/changeDP.py}
      \end{aside}

    \item Finally, in just a few sentences, give the big-O running time of your
      DP as a function of \(n\) and \Ident{amount}.

  \end{itemize}

\end{problem}

\begin{solution}

\end{solution}

\begin{problem}[Typesetting with Sho\TeX!, 60]

  This is an implementation problem using Python 3.  Researchers at the
  Sho'nuffian Institute of Technology (S.I.T.) are developing a new typesetting
  system called Sho\TeX.  One of Sho\TeX's features is that it formats input so
  that the text is  left-justified (all lines begin at the same left column)
  and the right margin is as ``even'' as possible. Moreover, Sho\TeX\ can
  format text for any column width (so that it fits on mobile device screens,
  computer screens, and other displays).  For example, consider the following
  text in an input file called \verb+mobydickExcerpt.txt+.

  \begin{aside}
    \inputminted{text}{./template/mobydickExcerpt.txt}
  \end{aside}

  Now, we run the \(\Ident{format}(\Ident{filename}, \Ident{columns}, k,
  \Ident{display})\) function (which you will write) to format this text. This
  function takes as input (1) the name of the input text file, (2) a positive
  integer indicating the number of columns in the output, (3) a penalty value
  \(k\) explained below, and (4) a Boolean \Ident{display} which determines
  whether or not to show the layout of the words in the solution. The program
  optimally formats the text to pack it into the given number of columns and it
  returns a single numerical value, which is the optimal total penalty.  The
  definition of optimality is explained below.

  \begin{aside}
    \mintinline{python}{>>> format("mobydickExcerpt.txt", 40, 3, True)}
    \tcblower
    \inputminted{text}{template/mobydick-example.txt}
    \mintinline{python}{259} \quad\textleftarrow{} This value was returned
  \end{aside}

  Here's where we do it with the display variable set to \verb+False+.

  \begin{aside}
    \mintinline{python}{>>> format("mobydickExcerpt.txt", 40, 3, True)}
    \tcblower
    \mintinline{python}{259}
  \end{aside}

  Here are the details:

  \begin{itemize}
    \item The text is assumed to be fixed-width font (so that every symbol has
      the same width).  Most likely your programming environment will
      automatically use a fixed-width font.  If not, you may need to set it to
      fixed-width.
    \item You should strip out \emph{all} white space, leaving only words. Some
      words will end with a punctuation symbol and that symbol simply is
      considered a letter that is part of the word that it's attached to. The
      word packing should pack words in sequence with exactly one space between
      each word.  (There is no special exception for more space after
      punctuation, since punctuation symbols are just considered characters in
      the words that contain them.)  Blank lines and all other extra white
      space will be removed!  Python's \Ident{split} method for strings makes
      this very easy.  (You can read more about it online.)
    \item The output text on each column may never exceed the specified column
      width.  (You should assume that the column width will always be at least
      as large as the length of the longest word in the input file. If the user
      provides a column width that is smaller than the length of some string in
      the file, your program may report an error message, behave badly, or
      crash.)
    \item The penalty function is of the form \(n^k\) where the user specifies
      the value of \(k\).  The penalty of a word packing is computed as
      follows: For every line in the output that contains text, compute the
      amount of slack on that line as the difference between the column width
      (40 in our example above) and the length of the text on that line.  Note
      that spaces between words contribute to the length of the line, but all
      spaces after the last word are counted as slack. Apply the penalty
      function to that slack.  This is called the \emph{line penalty}.  Add up
      all of the line penalties to get the total penalty.  In the example
      above, we used a penalty of \(n^3\) and the slacks on each of the six
      lines are 2, 2, 4, 3, 5, and 3 for a total penalty of
      \(2^3+2^3+4^3+3^3+5^3+3^3=259\).
    \item The \Ident{format} function should find a formatting that minimizes
      the penalty for the given text, column width, and penalty function. If
      the \Ident{display} variable is \Ident{True} it should \Ident{print} the
      formatted text (using the \Ident{print} function).  The function should
      always \mintinline{python}{return} a single number: the optimal total
      penalty.
    \item In some cases, there are multiple equally good solutions.  They will
      all have the same penalty, but may pack the words slightly differently.
      That's OK!
    \item Please test your code using our autotester for this problem. That
      autotester will provide part of your score for this problem, so you
      should be certain that the tests pass.
  \end{itemize}

  Your task is to write the \Ident{format} function in Python 3.  You will be
  provided the starter file \verb+format.py+.  Please add your code to that
  file. You should turn in a zip file that contains the following:
  \begin{enumerate}
    \item A text file \verb+mytest.txt+ that contains a test input that you
      constructed to test your code.  (Developing tests is a key skill for all
      computer scientists!)  You should develop this test input so that when
      you run your \Ident{format} DP with a particular number of columns and
      exponent \(k\) of your choosing, you know in advance what the optimal
      solution should be.  This will require that you think carefully about
      designing that text and the column width that you wish to use for testing
      your code.
    \item A copy of the provided \verb+tester.py+ file with one additional
      \mintinline{python}{assert} statement that tests on your
      \verb+mytest.txt+ file with some column width and exponent argument \(k\)
      of your choosing. Then, underneath that \mintinline{python}{assert}
      statement, add a Python comment that explains how you determined that the
      solution is correct; that is, how you designed the text and column width
      and exponent value \(k\) to know what the correct solution should be.
    \item A \verb+pdf+ document that contains the description of the recursive
      algorithm, the resulting DP, and a derivation of the running time of the
      DP. The running time may be a polynomial in the number of words, the
      total length of the words, and/or the number of columns in the formatted
      output.
    \item The \verb+format.py+ file  that contains your Python code added to
      our starter file.  Please include at least a few explanatory comments as
      a matter of good style.
  \end{enumerate}

  \begin{hint}
    The last word can go on a new line or not. The last two words can go on a
    new line or not. The last three words…
  \end{hint}

\end{problem}

\begin{solution}

\end{solution}

\end{document}
