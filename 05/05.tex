\documentclass{ks-pset}

\usepackage{ks-cs}

\title{Homework 5: Divide \& Conquer II}
\date{2022 March 2 (Wednesday)}
\author{}

\begin{document}

\begin{problem}[Order Statistics with 7!, 40]

  In class, we looked at the order statistics algorithm \Ident{Select} which
  takes as input an unsorted array \(A\) of length \(n\) and a positive integer
  \(i\) and returns the \(i\)-th smallest number in \(A\). Our first attempt
  was based on grouping \(A\) into groups of size 3.  That resulted in an
  algorithm whose running time was \(\AsymptoticO(n \log n)\).  But, using
  groups of \(5\) gave us a \(\AsymptoticO(n)\) algorithm.  Show that when we
  use groups of size \(7\), the algorithm still runs in time
  \(\AsymptoticO(n)\).  To do this, you'll need to derive and explain the
  recurrence relation for the algorithm and then explain carefully why it
  solves to \(\AsymptoticO(n)\).

\end{problem}

\begin{solution}

  \paragraph{Self-assessment}
\end{solution}

\begin{problem}[Pasadenium!, 60]

  This is a divide-and-conquer problem!  Researchers at the Pasadena Institute
  of Technology are experimenting with a super-strong new material called
  Pasadenium.  To test its strength, blocks of Pasadenium will be dropped from
  various floors of  a tall building with \(n\) floors.  A Pasadenium block is
  said to have \emph{least breaking floor} (LBF) \(k\) if it will break when
  dropped from floor \(k\) but will not break if dropped from any lower floor.
  The LBF is defined to be \(n+1\) if it will not break from floor \(n\) (the
  uppermost floor).

  Your task is to determine the LBF of a block of Pasadenium.  Since it's a
  very expensive material, you are given just two blocks.  Fortunately, any two
  blocks of Pasadenium have the same LBF. Unfortunately, once one of your
  blocks breaks, the resulting block (even if just cracked) is unusable for
  future testing.

  It is easy to see that if you only had one block, you would have to drop the
  block \(\AsymptoticO(n)\) times in the worst case to determine the LBF
  (successively drop the block from floor \(1\), floor \(2\), …, floor \(n\)).
  If at any step you were to skip up more floors, then you would risk
  prematurely breaking the block without knowing the exact floor on which it
  would have broken.

  In this problem we will determine how to best utilize two blocks to determine
  the LBF using the least number of drops. Rather than expressing the solution
  as ``Given a building with \(n\) floors, I can compute the LBF in at most
  \(f(n)\) drops'' we will instead express the solution \emph{inversely} as
  ``With \(d\) drops, I can compute the LBF of a building with as many as
  \(g(d)\) floors.''  Here's your task:

	\begin{subproblems}

    \item \label{part:gd} Given two blocks of Pasadenium and no more than \(d\)
      drops, for how tall a building can you guarantee finding the LBF?
      Describe your algorithm in clear and concise English and give a closed
      form function \(g(d)\) for the maximum height of the building (as a
      function of \(d\)) that your algorithm can handle.

    \item Use induction on \(d\) to prove that your algorithm uses at most
      \(d\) block drops for a building with \(g(d)\) floors.

    \item Dr. Devi is skeptical.  Write a short memo explaining why she should
      believe that your algorithm is optimal; that is, why \emph{any algorithm}
      that uses 2 blocks and at most \(d\) drops cannot guarantee determining
      the LBF for a building taller than the \(g(d)\) that you gave in part
      \ref{part:gd} of this problem.  A few sentences of intuitive explanation
      suffice here.  (If you wish, you're welcome to prove this
      rigorously---it's not too hard to turn the informal explanation into a
      formal proof.)

	\end{subproblems}

\end{problem}

\begin{solution}

  \begin{subproblems}

    \item

    \item

    \item

  \end{subproblems}

  \paragraph{Self-assessment}
\end{solution}

\end{document}
